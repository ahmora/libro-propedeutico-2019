\documentclass[oneside]{book}
\usepackage[spanish]{babel}
\usepackage[utf8]{inputenc}
\usepackage[T1]{fontenc}
\usepackage{graphicx}
\usepackage{amsmath}
\usepackage{subcaption}
\usepackage{amsthm}
\usepackage[
  top=3cm,
  bottom=3cm,
  left=2cm,
  right=2cm,
  heightrounded,
]{geometry}
\usepackage[svgnames]{xcolor} 

\pdfminorversion=5


\title{Ejemplo para el propedéutico de C. de la C.}
\author{Alejandro Hernández Mora \thanks{\texttt{alejandrohmora@ciencias.unam.mx}}}
\date{\today}

\theoremstyle{definition}
\newtheorem{teorema}{Teorema}[chapter] % reset theorem numbering for each chapter

\renewcommand{\labelitemi}{$\bullet$}
\begin{document}
%\maketitle

\begin{titlepage} % Suppresses headers and footers on the title page

  \raggedleft % Right align everything
  
  \vspace*{\baselineskip} % Whitespace at the top of the page
  
  %------------------------------------------------
  %	Author
  %------------------------------------------------
  
  {\Large El nombre de varios autores} % Author name
  
  \vspace*{0.167\textheight} % Whitespace before the title
  
  %------------------------------------------------
  %	Title and subtitle
  %------------------------------------------------
  
  \textbf{\LARGE Un libro escrito en conjunto:}\\[\baselineskip] % First title line
  
  {\textcolor{Red}{\Huge El curso propedéutico}}\\[\baselineskip] % Main title line which draws the focus of the reader
  
  {\Large \textit{Un poco sobre los estudiantes de nuevo ingreso}} % Subtitle
  
  \vfill % Whitespace between the titles and the publisher
  
  %------------------------------------------------
  %	Publisher
  %------------------------------------------------
  
  {\large Editor: Alejandro Hernández} % Publisher and logo
  
  \vspace*{3\baselineskip} % Whitespace at the bottom of the page

\end{titlepage}

\tableofcontents % indice de contenidos
\addcontentsline{toc}{chapter}{Índice}


\listoffigures % indice de figuras
\addcontentsline{toc}{chapter}{Índice de figuras} % para que aparezca en el indice de contenidos


\listoftables % indice de tablas
\addcontentsline{toc}{chapter}{Índice de tablas} % para que aparezca en el indice de contenidos

\chapter*{Introducción}
Este es un libro que hicimos todos los participantes de nuevo ingreso a la carrera
de \emph{Ciencias de la Computación} de la \emph{Facultad de ciencias} de la
\emph{Universidad Nacional Autónoma de México}.\\

En este libro podemos encontrar una breve biografía de cada uno de los estudiantes que cursaron
dicho curso en el \emph{Laboratorio de Ciencias de la Computación 1} en el edificio Tlahuizcalpan.\\

En el último día de dicho curso, tuvimos un susto de alerta, que creímos que era sísmica.\\

Este libro certifica que los alumnos que colaboraron con este proyeto, probablemente no sepan
nada de \LaTeX, sin mebargo tienen conocimiento de su existencia xD.

\chapter{Alejandro}

Plantilla de \large{\LaTeX.}
\section{Un poco sobre mi}
Me llamo Alejandro y soy egresado de Ciencias de la Computación en la Facultad de Ciencias :)

Mi trabajo de tesis estuvo basado en el artículo~\cite{Floodlight}.

Me gusta mucho leer los libros~\cite{torres,comunidad,retorno}

\subsection{Hobbies}
\begin{enumerate}
\item Hobbie 1
\item Hobbie 2
\item Hobbie 3
\end{enumerate}
Me gusta mucho la música, tocar la guitarra y cantar.

\begin{itemize}
\item Cosa 1
\item Cosa 2
\item Cosa 3
\end{itemize}


\section{Fórmulas}
El teorema de pitágoras existe desde la época de los griegos.

\begin{teorema}[De Pitágoras]
El cuadrado de la hipotenusa es igual a la suma de los cuadrados de los catetos.\\
$c^2 = a^2+b^2$
\end{teorema}

La chicharronera: $x= -b +-\frac{\sqrt{b^2-4ac}}{2a}$.

\begin{enumerate}
\item $c^2=a^2+b^2$
\item $x= -b +-\frac{\sqrt{b^2-4ac}}{2a}$
\end{enumerate}

\begin{itemize}
\item $c^2=a^2+b^2$
\item $x= -b +-\frac{\sqrt{b^2-4ac}}{2a}$ 
\end{itemize}

\begin{figure}[h]
  \centering
  \begin{subfigure}{0.45\textwidth}
    \includegraphics[width=0.8\textwidth]{pitagoras.png}
    \caption{La fórmula del teorema.}
    \label{fig:teo_pitagoras}
  \end{subfigure}
  \begin{subfigure}{0.45\textwidth}
    \includegraphics[width=1.0\textwidth]{pitagoras2.png}
    \caption{Demostración Gráfica del teorema.}
    \label{fig:grafica}
  \end{subfigure}
  \caption{Teorema de Pitágoras.}
\end{figure}

\section{Leyes de los Signos.}
\begin{table}[h]
  \centering
  \begin{tabular}{| c | c  | c |}
    \hline
    Operando & Operando & Resultado\\
    $+$  &  $+$  & $+$\\
    \hline
    $+$  &  $-$  & $-$\\
    \hline
    $-$  &  $+$  & $-$\\
    \hline
    $-$  &  $-$  & $+$\\
    \hline
  \end{tabular}
  \caption{Tabla con las leyes de los signos}
  \label{tab:leyes_signos}
\end{table}

\chapter{Jose Eduardo González Jasso}
\section{Mi descripcion.}
Me llamo Jose Eduardo voy a estudiar Ciencias de la computación en la Facultad de Ciencias. Elegí esta carrera pues me intersa aprendar a administrar Bases de Datos y porque a mi parecer en un futuro cercano las computadoras y en general los aparatos inteligentes formaran parte de la vida cotidiana de las personas, por lo cual es vital saber manejarlos profundamente.

\section{Mis libros favoritos}
\begin{enumerate}
\item El señor de los anillos: Las dos torres ~\cite{torres}
\item El señor de los anillos: El retorno del rey ~\cite{retorno}
\item El señor de los anillos: La comunidad del anillo ~\cite{comunidad}
  

\subsection{Hobbies.}
\begin{enumerate}
\item Escuchar música en especial de tipo Rock
\item Jugar videojuegos en mis horas libres
\item Pasear a mis perros en la noche
\end{enumerate}

\begin{figure}[h]
  \centering
  \includegraphics[scale=0.4]{IMG/13.jpg}
  \caption{Imagen de mi banda favorita}
  \label{fig:Soda}
\end{figure}

Podemos ver un ejemplo de una banda latinoamericana \textbf{\textbf{\emph{Figura}}}~\ref{fig:Soda}.

\newpage
Algunas formulas importantes en mi voda son:
\begin{itemize}
\item $c^2 = a^2 + b^2$
\item $x=-b \pm \frac{\sqrt{b^2-4ac}}{2a}$
\end{itemize}  
  
\section{Algo de matemáticas simples}

\begin{table}[h]
  \centering
  \begin{tabular} {| c c c |}
    \hline
    Operando & Operando & Resultado\\
    $+$ & $+$ & $+$\\
    $+$ & $-$ & $-$\\
    $-$ & $-$ & $-$\\
    $-$ & $+$ & $-$\\
    \hline
  \end{tabular}
  \caption{Leyes de los signos}
  \end{table}

\chapter{Alejandro Hernández Cano}

\section{Sobre mi}
Me llamo Alejandro y estoy cursando el propedeutico de Ciencias de la
Computacion en la Facultad de Ciencias de la UNAM.

\begin{figure}[h]
  \centering
  \includegraphics[scale=0.25]{3.png}
  \caption{Logo de la facultad de ciencias}
  \label{fig:fciencias}
\end{figure}

Podemos ver un ejemplo del logo de la facultad de ciencias en la
\emph{\textbf{figura}}~\ref{fig:fciencias}

\subsection{Hobbies}
Mis hobbies incluyen:
\begin{enumerate}
\item Escuchar música
\item Jugar videojuegos
\item Leer los libros ~\cite{torres, comunidad, retorno}
\end{enumerate}

\section{A profundidad}
Mucha gente se preguntará por qué decidí estudiar ciencias de la
computación. Este tema lo aboradré en las siguientes secciones
\subsection{Antecedentes}
Me gusta programar y me interesan las matemáticas.

\subsection{Las razones}
\begin{enumerate}
\item Me gusta programar
\item Me interesan las matemáticas
\end{enumerate}

\newpage
\section{Conocimientos}
\subsection{Leyes de los signos}
La \emph{\textbf{tabla}}~\ref{tab:signos} representa las leyes de los signos

\begin{table}[h]
  \centering
  \begin{tabular}{ c c c }
    Operando I & Operando II & Resultado\\
    \hline
    $+$ & $+$ & $+$\\
    $+$ & $-$ & $-$\\
    $-$ & $+$ & $-$\\
    $-$ & $-$ & $+$\\
  \end{tabular}
  \caption{Leyes de los signos}
  \label{tab:signos}
\end{table}

\subsection{Fórmulas importantes}
Algunas fórmulas importantes de mi vida son:
\begin{itemize}
\item $c^2 = a^2 + b^2$
\item $x = \frac{-b\pm\sqrt{b^2 - 4ac}}{2a}$
\end{itemize}

\section{Outro} 
\subsection{Agradecimientos}
Para el desarrollo de este libro, me gustaría hacer agradecimientos especiales
a los siguientes:
\begin{enumerate}
\item Mi familia
\item Mis maestros
\item Mis amigos
\item El lector
\end {enumerate}

\subsection{Palabras finales}
Este titulo tardó mucho tiempo en pensarse, desarrollarse, editarse e
inspeccionarse. Se ha llevado muchos tests de control para asegurarse de la
calidad del producto. Agradecemos su atención y ya.

\include{capitulo4}
\chapter{Shai Lèger Hernández}
	
	\begin{figure}[h]
		\centering
		\includegraphics[scale=1]{./IMG/5.png}     
		\caption{Imagen de prueba}
		\label{fig:prueba}
	\end{figure}

\section{Sobre mi}

     - Tengo 18 años, tengo nacionalidad Canadiense, mi apellido es francés, mi nombre es hebreo.
     
\section{Hobbies}

	\begin{itemize}
		\item Programación
		\item Piano
	\end{itemize}
	
%referencias
%El \textbf{Teorema}-\ref{fig:prueba}

\newpage
\chapter{Cosas de matemáticas}

\section{Formulas}

	\begin{itemize}
		\begin{huge}
			\item {$c^{2} = a^{2} + b^{2}$}\\
			\item {$x =\frac{-b \pm \sqrt{b^{2} - 4ac}}{2a}$}\\
		\end{huge}
	\end{itemize}


\section{Tabla de ley de los simbolos}

	\begin{table}[h]
		\centering
		
		\label{tab:leyes_signos}
		\begin{tabular}{ | c | c | c | }
			
			\hline
			Operando & Operando & Resultado\\
			\hline
			$+$ & $+$ & $+$\\
			$+$ & $-$ & $-$\\
			$-$ & $+$ & $-$\\
			$-$ & $-$ & $+$\\
			\hline
			
		\end{tabular}
		\caption{Tabla de signos}
	\end{table}
	
	\centering\emph{Plantilla para hacer una \emph{Tabla}~\ref{tab:leyes_signos} en \LaTeX}
		



\chapter{Alexia}

\section{un poco sobre mi}
\textbf{ADVERTENCIA. Lees esto a conciencia de que son tonterias.}

Me llaman Alexia, realmento solo aspiro a la felicidad, ya lo se, super cliche pero pues...me gusta, voy en la facultad de ciencias, porque, no es que sea especialmente buena, de hecho no lo soy, pero vale la pena seguir, es divertido :) ahhh y vengo de prepa 9, tome alguna clase de programacion c++ y pues me gusto, así que aquí estoy. 


Me gusta mucho leer los libros~\cite{torres,comunidad,retorno}

\subsection{hobbies}

\begin{enumerate}
\item cantar
\item respirar (hacer yoga y ballet)
\item tocar el piano
\end{enumerate}

\begin{figure}[h]
  \centering
  \includegraphics[scale=0.10]{IMG/6.jpg}
  \caption{my love}
  \label{fig:hamburguesa}
\end {figure}

Podemos ver una deliciosa hambueguesa llamada comunmente por mi: amor
\textbf{figura}~\ref{fig:hamburguesa}.

\begin{itemize}
\item $c^{-ba}$
\item $x=-b_a \pm \frac{5}{\sqrt{9}}$
 \end{itemize}

\begin{tabular}{|l | c | r|}
  \hline
  comida & yet & porcino \\
  \hline
  pepinillos & gorra & carpa \\
  \hline
  
\end{tabular}


\chapter{Karla}

\section{Un poco sobre mi}
Mi nombre es Karla Denia (\textbf{\emph{Figura de Poblado Denia en España}}~\ref{fig:Dénia}), tengo 18 años, estudie en la ENP6, me gusta programar desde la secundaria, pero fue hasta la carrera técnica en computación que decidí que ciencias  erami mejor opción y pues aquí estoy, tratando de no morir en el intento. Mi mayor sueño es viajar por todo el mundo.
\begin{figure}[h]
  \centering
  \includegraphics[scale=0.15]{IMG/7.jpg}
  \caption{Dénia, España}
  \label{fig:Dénia}
\end{figure}
\subsection{Hobbies}
\begin{itemize}
\item Jugar voleibol
\item Leer~\cite{torres,comunidad,retorno}, sobretodo en el transporte público
\item Escuchar música, mi instumento favorito es el violín.
\end{itemize}
\begin{figure}[h!]
  \centering
  {\includegraphics[scale=0.25]{IMG/7_2.jpg}}
  {\includegraphics[scale=0.29]{IMG/7_3.jpg}}
  {\includegraphics[scale=0.54]{IMG/7_4.jpg}}
  \caption{Hobbies}
\end{figure}

\include{capitulo8}

\chapter{Berenice Calvario Gonzalez}
\section{Un poco sobre mi}
Me llamo Berenice y soy estudiante de Ciencias de la Computación. 

\subsection{Hobbies}
\begin{enumerate}
\item Tocar el violín y el piano
  
  \begin{figure}[h]
    \centering
    \includegraphics[scale=0.02]{IMG/1.jpg}
    \caption{\small partitura} \label{fig:1}
  \end{figure}

  
      
\item Nadar

  \begin{figure}[h] 
    \centering
    \includegraphics[scale=0.5]{IMG/2.jpg}
    \caption{\small nadar} \label{fig:2}
  \end{figure}
  
\item
\end{enumerate}

\begin{table}[h]
  \centering
  \begin{tabular}{ c c c }
    \hline
    operando & operando & resultado\\
    $+$ & $+$ & $+$\\
    $+$ & $-$ & $-$\\
    $-$ & $+$ & $-$\\
    $-$ & $-$ & $+$\\
    \hline
  \end{tabular}
\end{table}

\chapter{Claudia Osorio Lopez}

\section{Algo Sobre mi:}
Me llamo Claudia, tengo 18 años. Vivo en Pedregal de San Nicolas. Estudie en CHH Sur y elegi la carrera de Ciencias de la Computacion en la Facultad de Ciencias en Ciudad Universitaria.

Me gusta leer~\cite{Floodlight,Tokuyama}


\subsection{Hobbies}
Entre mis hobbies estan:
\begin{enumerate}
  \item Escuchar Musica
  \item Leer
  \item Limpiar
  \item Caminar
\end{enumerate}

\begin{figure}[h]
  \centering
  \includegraphics[scale=0.2]{IMG/tacos.jpg}
  \caption{De mis comidad favoritas}
  \label{fig:tacos}
\end{figure}

\chapter{Alejandro Blancas Peralta}

\section{Algo Sobre mi:}
Me llamo Alejandro, tengo 18 años. Vivo en Cuautepec. Estudie en CHH Vallejo, estudiaré la carrera de Ciencias de la Computacion en la Facultad de Ciencias en Ciudad Universitaria.
 Leo:~\cite{comunidad,torres,retorno}

\subsection{Hobbies}
Entre mis hobbies estan:

\begin{enumerate}
  \item{Jugar Videojuegos
  \item{Dibujar
  \item{PvZx
\end{enumerate}

\begin{figure}[h]
  \centering
  \includegraphics[scale=0.2]{IMG/Girasol_.jpg}
  \caption{\small algo sobre eso} \label{fig:Girasol_}
\end{figure}
x
Podemos ver en la imagen anterior un hermoso girasol en la
\emph{\textbf{figura}}~\ref{fig:Girasol_}

\begin{table}[h]
  \centering
  \begin{tabular}{| c | c | c |}
    \hline
    Operando & Operando & Resultado\\\hline
    $+$ & $+$ & $+$\\\hline
    $+$ & $-$ & $-$\\\hline
    $-$ & $+$ & $-$\\\hline
    $-$ & $-$ & $+$\\\hline
    \hline
  \end{tabular}
\end{table}

\newpage
Algunas formulas importantes de la vida son:
\begin{itemize}
  \item $c^{2} = a^{2} + b^{2}$
  \item $X= -b \pm \frac{\sqrt{b^{2}-4ac}}{2a}$
    \end{itemize}

\include{capitulo12}
\include{capitulo13}
\chapter{Antonio Sebastian Dromundo Escobedo}

\section{Yo}
Soy Sebastian y soy estudiante de Ciencias de a computación en la Facultad de Ciencias :)

Solo he leido el primer libro del señor de los anillos ~\cite{comunidad}

\subsection{Hobbies}
\begin{enumerate}
\item Tocar la guitarra
\item Ver peliculas
\item Pasear
\end{enumerate}

3 cosas importantes para mi son

\begin{itemize}
\item Mi guitarra
\item Mi computadora
\item Mis fotos
\end{itemize}

\newpage
Just filling some space:
\begin{itemize}
\item $c^2 = a^2 + b^2$
\item $F_{e} = k \frac{q_{1} q_{2}}{r^2}$
\item $x= \frac{-b \pm \sqrt{b^2-4ac}}{2a}$
\end{itemize}

\section{This seems to be fun}
\begin{figure}[h]
  \centering
  \includegraphics[scale=0.5]{IMG/Fry.jpg}
  \caption{Wubba Lubba Dub Dub}
  \label{fig:fry}
\end{figure}

Not quite sure what to cite so I'm just letting a photo af myself \emph{life} en la \emph{\textbf{Figura~\ref{fig:fry}.}}

\begin{table}[h]
  \centering
  \begin{tabular}{| c  c  c  |}
    \hline
    Proposicion 1 & Proposicion 2 &1 y  2 \\\hline
    $+$ & $+$ & $+$\\\hline
    $+$ & $-$ & $-$\\\hline
    $-$ & $+$ & $-$\\\hline
    $-$ & $-$ & $+$\\\hline    
  \end{tabular}
  \caption{Tabla de verdad de Conjuncion}
  \label{tabla:afirmaciones}
\end{table}

En la \textbf{\emph{Tabla}}~\ref{tabla:afirmaciones} el signo afirmativo se refiere a verdadero y el negativo a falso.

\include{capitulo15}
\chapter{Diego Jardon}
\paragraph{
  13 de septiembre de 1999, yo nazco en un pueblo de estos de Guerrero pero me mudo a Michoacán por decisión de mi padre a los 4 años de edad, donde viví mi niñez y la mayor parte de la adolescencia, hoy (no precisamente hoy), me mudo a la Ciudad de México para estudiar en la Universidad Nacional Autonoma de Mexico.
}

\section{Hobbies:}

\begin{itemize}
\item Hacer Parkour
    \item Programar
    \item Caminar
    \item Leer \cite{comunidad}
\end{itemize}

\begin{figure}[h]
  \centering
  \includegraphics[scale=0.15]{IMG/13.jpg}
  \caption Parkour
\end{figure}

\chapter{José Luis García Santamaría}

\section{un poco sobre mi}
Acabo de entrar a la carrera de ciencias de la computacion, provengo del cch sur y elegi la carrera porque me parecio mas completa que la ingenieria en sistemas

\subsection{hobbies}
\begin{enumerate}

\item hobbie Nadar

  \begin{figure}[h]
    \centering
    \includegraphics[scale=0.5]{IMG/17.jpg}
  \end{figure}

\item hobbie Ver Youtube

  \begin{figure}[h]
    \centering
    \includegraphics[scale=0.5]{IMG/17.jpg}
    \caption{Viendo Youtube}
  \end{figure}
  
\item hobbie Me gusta escuchar musica

  \begin{figure}[h]
    \centering
    \includegraphics[scale=0.5]{IMG/17_2.jpg}
    \caption{Escuchando musica}
  \end{figure}
  
\end{enumerate}


\include{capitulo18}
\include{capitulo19}


\author{Miguel Angel Reyes}
\title{Mi bio}
\date{\today}

\renewcommand{\labelitemi}{$\bullet$}


\begin{document}

\maketitle
\tableofcontents
\addcontentsline{toc}{chapter}{Índice}

\listoffigures
\addcontentsline{toc}{chapter}{Índice de figuras}

\listoftables
\addcontentsline{toc}{chapter}{Índice de tablas}

\chapter*{Introducción}

Bueno aqui tendria que poner algo solo que no se me ocurre algo asi que voy a iniciar :(
\chapter{Miguel Angel}

\section{Un poco sobre mi}
Mi nombre completo es Miguel Angel Reyes Encarnacion, estoy estudiando la carrera de ciencias de la computacion

Mi trabajo de tesis estuvo basado en el artículo~\cite{Floodlight}.

Me gusta mucho leer los libros (estos no son a mi me gusta el terror mas stephen king)~\cite{torres,comunidad,retorno}

\subsection{Hobbies}
Mis hobbies son:
\begin{enumerate}
\item Leer
\item Jugar Videojuegos
\item Generalmente ve peliculas, la mayoria de terror
\end{enumerate}

\newpage
Algunas fórmulas importantes en mi vida son:
\begin{itemize}
\item $c^2 = a^2 + b^2$
\item $x=-b \pm \frac{\sqrt{b^2-4ac}}{2a}$
\end{itemize}

\section{Libro}
\begin{figure}[h]
  \centering
  \includegraphics[scale=0.5]{IMG/20.jpeg}
  \caption{Imagen de un Libro}
  \label{fig:Libro}
\end{figure}

Podemos ver un ejemplo del \emph{Teorema de Pitagoras} en la
\emph{\textbf{Figura}}~\ref{fig:Libro}.


\begin{table}[h]
  \centering
  \begin{tabular}{| c  c  c  |}
    \hline
    Operando & Operando & Resultado\\\hline
    $+$ & $+$ & $+$\\\hline
    $+$ & $-$ & $-$\\\hline
    $-$ & $+$ & $-$\\\hline
    $-$ & $-$ & $+$\\\hline    
  \end{tabular}
  \caption{Leyes de los signos}
  \label{tabla:leyes_signos}
\end{table}

Yo aprendí las leyes de los signos en la secundaria, tal como
se muestra en la \textbf{\emph{Tabla}}~\ref{tabla:leyes_signos}.

Bueno como veran copie todo, en realidad no se si puse todo bien pero espero que lo hayan disfrutado

\include{capitulo21}
\chapter{Azul}
Me llamo Azul y tengo el cabello rosa.

\section{Hobbies}

\begin{enumerate}
\item ver vídeos en YouTube
\item tocar el violín
\item aprender japonés
\end {enumerate}

\section{Información}

\begin{itemize}
\item Soy de la CDMX
\item Tengo 19 años
\item Entré por examen de selección
\item Me interesa el diseño web
\end{itemize}

\section{Formulas}
Algunas fórmulas importantes en mi vida son:
\begin{itemize}
\item $c^2=a^2+b^2$
\end{itemize}
  
\begin{table}[h]
  \centering
  \begin{tabular}{|c c c|}
    \hline
    Operando & Operando & Resultado\\
    $+$ & $+$ & $+$\\\hline
    $+$ & $-$ & $-$\\\hline
    $-$ & $+$ & $-$\\\hline
    $-$ & $-$ & $+$\\\hline
\end{tabular}
\end{table}

\section{Libros}
Los libros que me gustan son~\cite{salinger2001guardian, rice2014entrevista, shelley2008frankenstein, marias2011corazon}

\begin{figure}[h]
  \centering
  \includegraphics[scale=0.5]{img/22.png}
  \caption {violin}
  \label{fig:violin}
\end{figure}

Podemos ver un ejemplo de un violin en
\emph{\textbf {violín}}~\ref{fig:violin}

\chapter{Emiliano}
Me llamo Emiliano, tengo 18 años y vivo en Xochimilco

\section{Hobbies}

\begin{enumerate}
\item ver vídeos en YouTube
\item tocar la guitarra
\item jugar videojuegos
\end {enumerate}

\section{Información}

\begin{itemize}
\item Soy de la CDMX
\item Estudie en Prepa 5
\item Me interesa la programacion
\end{itemize}

\chapter{Mauricio Riva Palacio Orozco}

\section{Un poco sobre mi}
Me llamo Mauricio y soy estudiante de primer ingreso en ciencias de la computacion, me gusta escuchar música y salir 

\subsection{Hobbies}
Me gustan los deportes como el tennis, el futbol americano, etc. 

\section{Porque ciencias de la computacion?}
Porque es lo que me gusta y tengo muchas ganas de aprender

\subsection{Imagenes}
Una imagen de prueba

\begin{figure}[b]
  \centering
  \includegraphics[scale=0.5]{IMGA/linux.png}
  \caption{Imagen sobre linux}
\end{figure}

\include{capitulo25}
\chapter{Sebastián Alamina Ramírez}

\section{About me.}
Me llamo Sebastián, tengo 18 años y voy a estudiar Ciencias de la Computación en la Facultad de Ciencias de la UNAM (\textbf{Figura \ref{fig:26}}).

\begin{figure}[h]
\centering
\includegraphics[scale=1]{IMG/26.jpg}
\caption{Facultad de Ciencias, UNAM.}
\label{fig:26}
\end{figure}

\chapter{Miguel Ángel Ordóñez}

\section{Yo}
Mi nombre es Miguel Ángel Ordóñez Silis. Nací el 5 de agosto de 1994 en la Ciudad de México.

\subsection{Hobbies}

Estos son algunos de los hobbies que tengo: 

\begin{itemize}
\item Videojuegos
\item Preparar café
\item Leer
\item Andar en bicicleta
\end{itemize}

\begin{figure}[h]
  \centering
  \includegraphics[scale=.5]{IMG27/27.jpg}
  \caption{Specialized Crosstrail Mech}
  \label{fig:bicicleta}
\end{figure}

\subsection{Libros}

Actualmente estoy leyendo \emph{Las cinco ecuaciones que cambiaron al mundo.~\cite{comunidad}}

\include{capitulo28}
\include{capitulo29}
\include{capitulo30}
\include{capitulo31}
\include{capitulo32}




\bibliographystyle{acm}
\bibliography{bibliografia}
\end{document}


